\documentclass{article}
\usepackage[left=2cm,right=2cm,top=2cm,bottom=2cm]{geometry}
\usepackage[utf8]{inputenc}
\usepackage{amssymb}
\usepackage{amsmath}
\usepackage{amsthm}

\title{Design-Connectivity-Maher-2019}
\author{Maher \and Yann \and Alain}
\date{April 2019}

\begin{document}

\maketitle

\newtheorem{theorem}{Theorem}
\newtheorem{result}[theorem]{Result}
\newtheorem{corollary}[theorem]{Corollary}

\newcommand{\Nat}{\mathbb{N}}
\newcommand{\PosInt}{\mathbb{N}^*}
\newcommand{\Forb}{\mathcal{F}}
\newcommand{\Lang}[1]{\mathcal{L}_{#1}}
\newcommand{\Hamming}[2]{H(#1,#2)}



\section{Introduction}
{\it A faire par Maher}

\subsection*{Definitions}

Let $n \in \Nat, n \geq 2$. \\
Let $\Sigma$ be an alphabet, $|\Sigma| \geq 2$. (We are especially interested in the case $\Sigma = \{A, U, G, C\}$). \\
Let $\Forb$ be the set of forbidden motifs. \\
Let $\Lang{\Forb}$ be the set of words in $\Sigma^n$ that do not contain any motif in $\Forb$. \\
Let $H(w,w')$ be the Hamming distance between two words $w$, $w'$ in $\Sigma^n$.

\subsection*{General problem}

Input: $\Forb$ a set of forbidden motifs, $\delta: \Lang{\Forb} \rightarrow \Lang{\Forb}$ a neighborhood function on $\Lang{\Forb}$ \\
Question: The graph $G = (\Lang{\Forb}, \delta)$ is strongly connected.

\section{Results}
\subsection{With the $k$-Hamming neighborhood}

Given $k \in \PosInt$, we define $\delta_k$ {\it the k-Hamming neighborhood} as follows:
$$\forall w \in \Lang{\Forb},  \delta_k(w) = \{w'\in \mathcal{L}_{\Forb} \mid \Hamming{w}{w'}\le k\}. $$
With $k = n$, any $w \in \Lang{\Forb}$ can be changed into any other $w' \in \Lang{\Forb}$ in one step. Hence $G = (\Lang{\Forb}, \delta_n)$ is always strongly connected. \\
Thus with the $k$-Hamming neighborhood a variant of the general problem can be considered: \\
\newline
Input: $\Forb$ a set of forbidden motifs \\
Question: the minimal $k \in \PosInt$ such that the graph $G = (\Lang{\Forb}, \delta_k)$ is strongly connected.


\subsubsection{One motif}

Consider the case where $\Forb$ contains a single motif: $\Forb = \{f\}$. \\
Then $k = 1$ is sufficient to guarantee strong connectivity.
\begin{result}
	$\forall f \in \Sigma^+$, $G = (\Lang{\{f\}}, \delta_1)$ is strongly connected.
\end{result}
\begin{proof}
	Let $w$ and $w'$ be two words in $\Lang{\{f\}}$ (of length $n$). \\
	As $f \neq \epsilon$, $f$ can be written letter by letter as follows: $f = f_1 ... f_{|f|}$. \\
	Since $|\Sigma| \geq 2$, let $a \in \Sigma$ such that $a \neq f_1$. \\
	We show that there is a path from $w$ to $a^n$. \\
	To do so, from left to right we replace each letter in $w$ by $a$ (or keep it the same if already an $a$). \\
	Formally, from $w = w_1 ... w_n$ we define the sequence $(u_i)_{0 \leq i \leq n}$ of intermediate words:
	$$ \forall 0 \leq i \leq n, u_i = a^i w_{i+1} ... w_n.$$
	Then:
	\begin{itemize}
		\item $\forall 0 \leq i \leq n - 1, H(u_i, u_{i+1}) \leq 1,$
		\item for every $i$ in $[1..n]$ we must prove that $u_i$ is in $\Lang{\{f\}}$. By contradiction, suppose that $f$ appears in a $u_i$. Let $j$ be the position in $u_i$ of the leftmost letter of this occurrence of $f$.
		\begin{itemize}
			\item if $j \leq i:$ then the leftmost letter of $f$ would be $a$, which is not by definition of $a$.
			\item if $j > i:$ then this occurrence of $f$ would be a factor of $w$, which it cannot be since $w \in \Lang{\{f\}}.$
		\end{itemize}
	Contradiction. Hence every $u_i$ is in $\Lang{\{f\}}$.
	\end{itemize}
	This proves that there is a path from $w$ to $a^n$ with $\delta_1$ as the neighborhood function. \\
	The same can be done to obtain a path from $w'$ to $a^n$. \\
	Finally, since $\delta_1$ is symmetric this gives a path from $w$ to $w'$ and vice-versa.	
\end{proof}

In addition to show that $G = (\Lang{\{f\}}, \delta_1)$ is strongly connected, this proof gives us $2n$ as an upper bound to the diameter of $G$.

We could have tried to prove Result 1 by induction on $H(w,w')$ instead. But it is unclear to what extent such an induction would be feasible (at least for now). Consider the following example with $\Sigma = \{A,U\}:$
$$ \Forb = \{AUA\}, u=AAAA, v=AUUA.$$
There is no way to replace one non-extremal $A$ of $u$ with $U$ without getting an occurrence of $AUA$. Hence there is no path from $u$ to $v$ in $G$ with decreasing Hamming distance, even though $u$ and $v$ are connected according to Result 1. The same idea gives counter-examples of arbitrary Hamming distance:
$$\forall i \in \PosInt, i \geq 2, with: \Forb = \{AUA\}, u_i=A^{i+2}, v_i=A U^i A,$$
$$then: H(u_i,v_i) = i.$$
These examples heavily rely on the fact that $|\Sigma|=2$. There might be a way to get around this issue when $|\Sigma| \geq 3$ and find paths with non-increasing Hamming distance, but this would have to be looked at.

\subsubsection{Two or more motifs}

The idea from the proof of Result 1 could be used again to treat the cases when there is an available letter to do the same trick.
\begin{result}
	Let $F = \{f_1,...,f_k\}$ be the set of forbidden motifs.
	\begin{itemize}
		\item If there exists $a \in \Sigma$ that is the first letter of no $f_i$, then $G = (\Lang{\Forb}, \delta_1)$ is strongly connected.
		\item Same result if there exists $a \in \Sigma$ that is the last letter of no $f_i$.
	\end{itemize}
\end{result}
\noindent
This tells us that we need at least $|\Sigma|$ forbidden motifs to obtain a disconnected graph with $\delta_1$. Indeed if there are less than $|\Sigma|$ motifs, then we know that at least one letter is not the first letter of any forbidden motif.

\begin{corollary}
	If $|\Forb| < |\Sigma|$, then $G = (\Lang{\Forb}, \delta_1)$ is strongly connected.
\end{corollary}
\noindent
An example with $|\Sigma|$ words that gives a disconnected graph with $\delta_1$ is the following:
$$With: \Sigma = \{a_1, a_2,...,a_k\}, let: \Forb = \{a_1 a_2, a_2 a_1, a_3,...,a_k\}.$$
Then the only two allowed words are $a_1^n$ and $a_2^n$, and there is no way to go from one word to the other.



\end{document}
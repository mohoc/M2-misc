\documentclass{article}
\usepackage[margin=3cm]{geometry}
\usepackage[utf8]{inputenc}
\usepackage{amssymb}
\usepackage{amsmath}
\usepackage{amsthm}
\usepackage{MnSymbol}


\usepackage{todonotes}
\newcommand{\yp}[1]{\todo[color=blue!20]{\sf Yann: #1}}
\newcommand{\ypi}[1]{\todo[color=blue!20, inline]{\sf Yann: #1}}

\title{Design-Connectivity-Maher-2019}
\author{Maher \and Yann \and Alain}
\date{April 2019}

\begin{document}

\maketitle

\newtheorem{theorem}{Theorem}
\newtheorem{proposition}[theorem]{Proposition}
\newtheorem{corollary}[theorem]{Corollary}
\newtheorem{definition}[theorem]{Definition}
\newtheorem{lemma}[theorem]{Lemma}
\newtheorem{remark}[theorem]{Remark}
\newtheorem{result}[theorem]{Result}
\newtheorem{example}[theorem]{Example}
\newtheorem*{idea*}{Idea}



\newcommand{\Nat}{\mathbb{N}}
\newcommand{\PosInt}{\mathbb{N}^*}
\newcommand{\Forb}{\mathcal{F}}
\newcommand{\ForbEqv}{\widetilde{\Forb}}
\newcommand{\Lang}[1]{\mathcal{L}_{#1}}
\newcommand{\LFn}{\Lang{\Forb, n}}
\newcommand{\Compl}[1]{\complement {#1}}
\newcommand{\Hamming}[2]{H(#1,#2)}
\newcommand{\EqDef}{\mathrel{\stackrel{\makebox[0pt]{\mbox{\normalfont\tiny def}}}{=}}}
\newcommand{\DeBruijn}[1]{\mathcal{DB}_{#1}}
	

\section{Introduction}
{\it A faire par Maher}

\subsection*{Definitions}

Let $n \in \Nat, n \geq 2$. \\
Let $\Sigma$ be an alphabet, $|\Sigma| \geq 2$. (We are especially interested in the case $\Sigma = \{A, U, G, C\}$). \\
Let $\Forb$ be the set of forbidden motifs. \\
Let $\LFn$ be the set of words in $\Sigma^n$ that do not contain any motif in $\Forb$. \\
Let $\Lang{\Forb}$ be the set of words in $\Sigma^*$ that do not contain any motif in $\Forb$. \\
Let $H(w,w')$ be the Hamming distance between two words $w$, $w'$ in $\Sigma^n$. \\

\noindent
Let $m(\Forb)$ $\EqDef$ $max_{f in \Forb} |f|$. \\
\textbf{We assume that $n \geq m(\Forb)$.} (Otherwise some forbidden motifs would never be problematic).\\
Hence any forbidden motif $f$ in $\Forb$ of length $l < m(\Forb)$ is equivalent to the set: $\bigcup_{i=0}^{n-l} \Sigma^i f \Sigma^{n-l-i}$ of forbidden motifs, all of legnth $m(F)$. \\
Thus we define $\ForbEqv$ the set of forbidden motifs - all of length $m(\Forb)$ - equivalent to $\Forb$.

\subsection*{General problem}

Input: $n \geq 2$, $\Forb$ a set of forbidden motifs, $\delta: \LFn \rightarrow \LFn$ a neighborhood function on $\LFn$ \\
Question: The graph $G = (\LFn, \delta)$ is strongly connected.

\section{Results}

\subsection{With the $k$-Hamming neighborhood}
\begin{definition}
	Given $k \in \PosInt$, we define $\delta_k$ the k-Hamming neighborhood as follows:
	$$\forall w \in \LFn,  \delta_k(w) = \{w'\in \mathcal{L}_{\Forb} \mid \Hamming{w}{w'}\le k\}. $$
\end{definition}

With $k = n$, any $w \in \LFn$ can be changed into any other $w' \in \LFn$ in one step. Hence $G = (\LFn, \delta_n)$ is always strongly connected. Thus with the $k$-Hamming neighborhood a variant of the general problem can be considered:

\subsubsection*{General problem with the $k$-Hamming neighborhood}

Input: $n \geq 2$, $\Forb$ a set of forbidden motifs \\
Question: the minimal $k \in \PosInt$ such that the graph $G_{\Forb, n, k} \EqDef (\LFn, \delta_k)$ is strongly connected.

\begin{remark}
	Since $\delta_k$ is symmetric for any $1 \leq k \leq n$, $G_{\Forb, n, k}$ is connected iff it is strongly connected. Thus we will use "connected" and "strongly connected" interchangeably when considering $k$-Hamming neighborhoods.
\end{remark}

\subsubsection{One motif}

Consider the case where $\Forb$ contains a single motif: $\Forb = \{f\}$. \\
Then $k = 1$ is sufficient to guarantee strong connectivity.
\begin{result}
	$\forall f \in \Sigma^+$, $G_{\{f\}, n, 1}$ is strongly connected.
\end{result}
\begin{proof}
	Let $w$ and $w'$ be two words in $\Lang{\{f\}}$ (of length $n$). \\
	As $f \neq \epsilon$, $f$ can be written letter by letter as follows: $f = f_1 ... f_{|f|}$. \\
	Since $|\Sigma| \geq 2$, let $a \in \Sigma$ such that $a \neq f_1$. \\
	We show that there is a path from $w$ to $a^n$. \\
	To do so, from left to right we replace each letter in $w$ by $a$ (or keep it the same if already an $a$). \\
	Formally, from $w = w_1 ... w_n$ we define the sequence $(u_i)_{0 \leq i \leq n}$ of intermediate words:
	$$ \forall 0 \leq i \leq n, u_i = a^i w_{i+1} ... w_n.$$
	Then:
	\begin{itemize}
		\item $\forall 0 \leq i \leq n - 1, H(u_i, u_{i+1}) \leq 1,$
		\item for every $i$ in $[1..n]$ we must prove that $u_i$ is in $\Lang{\{f\}}$. By contradiction, suppose that $f$ appears in a $u_i$. Let $j$ be the position in $u_i$ of the leftmost letter of this occurrence of $f$.
		\begin{itemize}
			\item if $j \leq i:$ then the leftmost letter of $f$ would be $a$, which is not by definition of $a$.
			\item if $j > i:$ then this occurrence of $f$ would be a factor of $w$, which it cannot be since $w \in \Lang{\{f\}}.$
		\end{itemize}
	Contradiction. Hence every $u_i$ is in $\Lang{\{f\}}$.
	\end{itemize}
	This proves that there is a path from $w$ to $a^n$ with $\delta_1$ as the neighborhood function. \\
	The same can be done to obtain a path from $w'$ to $a^n$. \\
	Finally, since $\delta_1$ is symmetric this gives a path from $w$ to $w'$ and vice-versa.	
\end{proof}

In addition to show that $G = (\Lang{\{f\}}, \delta_1)$ is strongly connected, this proof gives us $2n$ as an upper bound to the diameter of $G$.

\begin{remark}
	We could have tried to prove Result 1 by induction on $H(w,w')$ instead. But it is unclear to what extent such an induction would be feasible (at least for now). Consider the following example with $\Sigma = \{A,U\}:$
	$$ \Forb = \{AUA\}, u=AAAA, v=AUUA.$$
	There is no way to replace one non-extremal $A$ of $u$ with $U$ without getting an occurrence of $AUA$. Hence there is no path from $u$ to $v$ in $G$ with decreasing Hamming distance, even though $u$ and $v$ are connected according to Result 1. The same idea gives counter-examples of arbitrary Hamming distance:
	$$\forall i \in \PosInt, i \geq 2, \text{with: }\Forb = \{AUA\}, u_i=A^{i+2}, v_i=A U^i A,$$
	$$\text{then: } H(u_i,v_i) = i.$$
	These examples heavily rely on the fact that $|\Sigma|=2$. There might be a way to get around this issue when $|\Sigma| \geq 3$ and find paths with non-increasing Hamming distance, but this would have to be looked at.\ypi{Another possible direction is to choose another candidate for an intermediate word, \emph{i.e.} instead of showing connectivity as $$w \leftrightarrow A^n \leftrightarrow w',$$ find $u(w,w')$, depending on $w$ and $w'$ such that $$w \leftrightarrow u(w,w') \leftrightarrow w'$$}
\end{remark}

\begin{idea*}
	Give an arbitrary order on the letters in $\Sigma$ and take as the representative of each connected component their smallest element w.r.t the lexical order?
\end{idea*}

\subsubsection{Two or more motifs}

The idea from the proof of Result 1 could be used again to treat the cases when there is an available letter to do the same trick.
\begin{result}
	Let $F$ be the set of forbidden motifs.
	\begin{itemize}
		\item If there exists $a \in \Sigma$ such that: $\forall f\in \mathcal{F}, f[1] \neq a, $, then $G = (\LFn, \delta_1)$ is strongly connected.
		\item Same result if there exists $a \in \Sigma$ such that: $\forall f\in \mathcal{F}, f[|f|] \neq a$.
	\end{itemize}
\end{result}
\noindent
This tells us that we need at least $|\Sigma|$ forbidden motifs to obtain a disconnected graph with $\delta_1$. Indeed if there are less than $|\Sigma|$ motifs, then we know that at least one letter is not the first letter of any forbidden motif.

\begin{corollary}
	If $|\Forb| < |\Sigma|$, then $G_{\Forb, n, 1}$ is strongly connected.
\end{corollary}
\noindent
An example with $|\Sigma|$ words that gives a disconnected graph with $\delta_1$ is the following:
$$with: \Sigma = \{a_1, a_2,...,a_k\}, let: \Forb = \{a_1 a_2, a_2 a_1, a_3,...,a_k\}.$$
Then the only two allowed words are $a_1^n$ and $a_2^n$, and there is no way to go from one word to the other. \\

\noindent
\underline{\textbf{Case $k = n-1$, $|\Sigma| = 2$}} \\

\begin{result}
	If $k = n-1$ and $|\Sigma| = 2$, then: \\
	$u$ and $v$ are disconnected in $G = (\LFn, \delta_{n-1})$ iff:
	\begin{itemize}
		\item $u$ is the opposite word of $v$ in $\Sigma^n$,
		\item $\LFn = \{u,v\}$.
	\end{itemize}
\end{result}
\begin{proof}

	$(\Leftarrow)$ As $u$ and $v$ are opposite, $\Hamming{u}{v} = n$. Hence $u$ and $v$ are not neighbors and they are the only elements in $G$. \\
	
	$(\Rightarrow)$
	\begin{itemize}
		\item With $|\Sigma| = 2$ the only word in $\Sigma^n$ at Hamming distance greater than $n-1$ from $u$ is its opposite word.
		\item By contradiction: if any other word $w$ were in $\LFn$, then $w$ would be a $\delta_{n-1}$-neighbor of both $u$ and $v$, and thus $u$ and $v$ would be connected.
	\end{itemize}
\end{proof}

Then the only possible disconnected graph here is with two nodes that are opposite sequences, which is very restrictive. This makes for a simple example to study the impact of $\Forb$ on the strong connectivity of $G_{\Forb, n, n-1}$.

\noindent
--------------------------------

\begin{proposition}
	If $P_1$ and $P_2$ are in $\Lang{\Forb, |P_1|}$ and disconnected in $G_{\Forb, |P_1|, k}$ (for some $k \in [1..|P_1|]$), then:
	$\forall (S_1, S_2)$ couple of words of the same length: ($P_1 S_1$ and $P_2 S_2$ are in $\Lang{\Forb, |P_1|}$) $\Rightarrow$ ($P_1 S_1$ and $P_2 S_2$ are disconnected in $G_{\Forb, |P_1 S_1|, k}$).
\end{proposition}
\noindent
In other words, if we find two words $P_1, P_2$ that are disconnected in $G_{\Forb, i, k}$ for some $i \in \PosInt$, then any couple of words $P_1 S_1, P_2 S_2$ in $G_{\Forb, j, k}$ (for some $j > i$) that have them as their prefixes are disconnected as well.

\begin{proof}
	Let $P_1$ and $P_2$ be two words in $\Lang{\Forb, |P_1|}$ that are disconnected in $G_{\Forb, |P_1|, k}$. \\
	By contradiction: if there exist $S_1$, $S_2$ such that $P_1 S_1$ and $P_2 S_2$ are connected in $G_{\Forb, |P_1 S_1|, k}$, \\
	then there is a path $(P_1 S_1 = u_0) \rightarrow u_1 \rightarrow [...] \rightarrow u_i \rightarrow (u_{i+1} = P_2 S_2)$ in $G_{\Forb, |P_1 S_1|, k}$. \\
	We know then that: $\forall 0 \leq j \leq i, \Hamming{u_j}{u_{j+1}} \leq k$. \\
	For $0 \leq j \leq i+1$ let $P'_j$ be the prefix of length $|P_1|$ in $u_j$. \\
	Since we take the prefixes, we still have: $\forall 0 \leq j \leq i, \Hamming{P'_j}{P'_{j+1}} \leq k$. \\
	Hence $(P_1 = P'_0) \rightarrow P'_1 \rightarrow [...] \rightarrow P'_i \rightarrow (P'_{i+1} = P_2)$ is a valid path in $G_{\Forb, |P_1|, k}$. \\
	Hence $P_1$ and $P_2$ would be connected in $G_{\Forb, |P_1|, k}$. \\
	Contradiction.
\end{proof}

\begin{idea*}
	Use a De Bruijn graph to know when a word is a prefix of arbitrary long allowed words. See $Remark$ $11.$ .
\end{idea*}

\subsection{De Bruijn graphs}

\subsubsection{Properties}

\begin{definition}
	Given $\Forb$, we define $\DeBruijn{\Forb}$ the De Bruijn graph of $\Forb$ the following way:
	\begin{itemize}
		\item Vertices: $\Compl{\ForbEqv}$ the allowed substrings of length $m(\Forb)$.
		\item Edges: if $u \in \Sigma^{m(\Forb)-1}$, if $a,b \in \Sigma$, then: there is an edge from $au$ to $ub$ iff $au$ and $ub$ are both in $\Compl{\ForbEqv}$.
	\end{itemize}
\end{definition}

\begin{proposition}
	Let $w = w_1...w_n$ be a word of length $n$. Then: 
	\begin{itemize}
		\item[] $w \in \LFn$ iff $w_1...w_{m(F)} \rightarrow w_2...w_{m(F) + 1} \rightarrow [...] \rightarrow w_{n - m(F) + 1}...w_n$ is a valid path in $\DeBruijn{\Forb}$.
	\end{itemize}
\end{proposition}
\begin{proof}
	$w$ is an allowed word iff every factor of length $m(\Forb)$ in $w$ is allowed.
\end{proof}

\begin{remark}
	There are arbitrarily long allowed words iff there is a cycle in $\DeBruijn{\Forb}$.
\end{remark}

\begin{result}
	Let $u_1 \rightarrow [...] \rightarrow u_i$ be a path in $\DeBruijn{\Forb}$ ($i \geq 3$). \\
	If $u_i$ is a neighbor of $u_1$, then ($u_2, [...], u_{i-1}$) is a cycle in $\DeBruijn{\Forb}$.
\end{result}
\noindent
In other words: if there is a shortcut to a path in $\DeBruijn{\Forb}$, then the intermediate elements form a cycle.
\begin{proof}
	Since $u_i$ is a a neighbor of $u_1$ in $\DeBruijn{\Forb}$, we can write: $u_1 = av$ and $u_i = vb$ for some $v \in \Sigma^{m(\Forb) - 1}$ and $a, b \in \Sigma$. \\
	But then we know as well that: $u_2 = vc$ and $u_{i-1} = dv$ for some $a, b \in \Sigma$. \\
	Hence $u_2$ is a neighbor of $u_{i-1}$ and ($u_2, [...], u_{i-1}$) forms a cycle in $\DeBruijn{\Forb}$.
\end{proof}

\begin{idea*}
	How to interpret Hamming edits with paths in $\DeBruijn{\Forb}$?
\end{idea*}

\begin{definition}
	We define $\DeBruijn{\Forb, n}$ the graph obtained by removing all the connected components in $\DeBruijn{\Forb}$ that do not encode any word of length $n$.
\end{definition}
\noindent
The goal of this notion is to only keep the meaningful part in $\DeBruijn{\Forb}$ that generates the allowed words of length $n$. This is the same as removing all the connected components that have no path of length $\geq n - m(\Forb)$.

\begin{remark}
	For any $n \geq m(\Forb)$, $\DeBruijn{\Forb}$ and $\DeBruijn{\Forb, n}$ have exactly the same paths of length $n - m(\Forb)$.
\end{remark}

\begin{lemma}
	If we follow the same sequence of letters $a_1, a_2, [...], a_j$ from two distinct sequences $u, v$ in $\DeBruijn{\Forb}$, if $j \geq m(\Forb)$, then the two subsequent paths have merged at some index $i \leq m(\Forb)$.
\end{lemma}
\begin{proof}
	After $m(\Forb)$ steps the resulting word is $a_1...a_{m(\Forb)}$ in both paths, so the paths merged either at index $m(\Forb)$ or at a smaller index.
\end{proof}

\subsubsection{Applications}

\noindent
\underline{\textbf{Back to the case $k = n-1$, $|\Sigma| = 2$}} \\

\begin{result}
	With $|\Sigma| = 2$ (for instance $\Sigma = \{A, C\}$): $G_{\Forb, n, n-1}$ is disconnected iff $\DeBruijn{\Forb, n}$ is either:
	\begin{itemize}
		\item[(i)] A...A$\lcirclearrowleft$, C...C$\lcirclearrowleft$
		\item[(ii)] ACA... $\leftrightarrow$ CAC...
		\item[(iii)] two "opposite" paths of length $n - m(\Forb)$ with no connection: $u_1 \rightarrow [...] \rightarrow u_{n-m(\Forb)+1}$, $\overline{u_1} \rightarrow [...] \rightarrow \overline{u_{n-m(\Forb)+1}}$, \\
		where $\overline{u_i}$ is the opposite word of $u_i$.
	\end{itemize}
\end{result}
\begin{proof}
	$(\Leftarrow)$ All the three cases for $\DeBruijn{\Forb, n}$ imply that there are exactly two paths of length $n - m(\Forb)$ in $\DeBruijn{\Forb}$. By $Proposition$ $10.$, we deduce that there are only two words in $\LFn$ and they are opposite, which by $Result$ $7.$ means that $G_{\Forb, n, n-1}$ is disconnected. \\
	
	$(\Rightarrow)$ We know from $Result$ $7.$ that the only way to have $G_{\Forb, n, n-1}$ disconnected is to  have exactly two vertices and that they are opposite to each other. Using $Proposition$ $9.$, this means that we must exactly have two paths of length $n - m(\Forb)$ in $\DeBruijn{\Forb}$. \\
	Now we show that $\DeBruijn{\Forb, n}$ can only be of one the three proposed forms.
	\begin{itemize}
		\item If there is a 1-cycle in $\DeBruijn{\Forb, n}$: \\
		then this 1-cycle is either A...A$\lcirclearrowleft$ or C...C$\lcirclearrowleft$. In any case the other one must be included as well in order to include the path for the opposite word. There are already two paths of length $n - m(\Forb)$ in the graph [A...A$\lcirclearrowleft$, C...C$\lcirclearrowleft$] and appending any element to these components would add another path of length $n - m(\Forb)$, which we do not want. Thus the only graph $\DeBruijn{\Forb, n}$ that can have a 1-cycle is [A...A$\lcirclearrowleft$, C...C$\lcirclearrowleft$], which is case (i).
		\item If there is a 2-cycle in $\DeBruijn{\Forb, n}$: \\
		then this 2-cycle can only be ACA... $\leftrightarrow$ CAC..., which already encodes two words of length $n$. Again, appending any element to this component would add another allowed word of length $n$, so the only graph $\DeBruijn{\Forb, n}$ that can have a 2-cycle is [ACA... $\leftrightarrow$ CAC...], which is case (ii).
		\item If there is a (3+)-cycle in $\DeBruijn{\Forb, n}$: \\
		then there would be at least three allowed words of length $n$ (we obtain them by starting from a different element of the cycle as the prefix and by going through the cycle). Since we only want two allowed words of length $n$, $\DeBruijn{\Forb, n}$ cannot have a (3+)-cycle.
		\item If there is no cycle in $\DeBruijn{\Forb, n}$: \\
		TODO
	\end{itemize}
\end{proof}

\subsection{Algorithmic aspects}


\end{document}